\documentclass[10pt]{article}
 
\usepackage[latin1]{inputenc}
\usepackage[T1]{fontenc}  
\usepackage{musixtex}
\usepackage[frenchb]{babel}

\begin{document}

\normalmusicsize 

\begin{music}

\instrumentnumber{2} % 2 instruments

\setstaffs 1{2} % instrument 1 (en bas) : 2 portées

\setclef{1}{60} % clef de fa (6) en 1, clef de sol (0) en 2
\generalmeter{\meterfrac{4}{4}} % mesure 4/4
\setname 1{piano} %
\setname 2{chant} %
\parindent 10mm % pour éviter la collision de piano avec l'accolade
\startextract
% {{bleu|1re mesure}}
\Notes 
\ha J | % chgt portée, même instr
\zhu{c e}\hu g & % chgt instr ; pas d'espace entre } et \hu
\islurd0c\ibu0d0\qb0{c c c}\tslur0d\tbu0\qb0d 
\enotes % assure l'alignement
%
\Notes 
\ha N | % chgt portée, même instr
\zhu{g i}\hu k & % chgt instr
\qa{e d} 
\enotes 
\bar % après toutes les notes de la mesure, pour toutes les voix
% {{bleu|2e mesure}}
\Notes
\qa J | 
\zqu{c e}\qu g &
\islurd0c\ibu0d0\qb0{c e} 
\enotes
%
\Notes
\qa N |
\zqu{g i}\qu k &
\qb0{d}\tslur0d\tbu0\qb0d
\enotes
%
\Notes
\ha J | 
\zhu{c e}\hu g &
\ha c 
\enotes
%
\endextract

\end{music}

\end{document}